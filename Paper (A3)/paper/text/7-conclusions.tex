\section{Conclusion}
\label{sec:conclusions}

	% Context and problem definition
	Lightweight manycores brought together concepts of parallel and
	distributed systems into a single die to deliver high-performance and
	energy efficiency.  Nevertheless, architectural intricacies and the
	absence of \apis that embrace programmability and portability
	make software development an arduous task, specifically because
	current solutions are hardware-dependent and/or vendor-specific \apis.

	% Goals and contribution
	To unite programmability and portability for \lws,
	we proposed \lwmpi, a lightweight and portable \mpi implementation on
	top of a \posix-compliant distributed \os that targets this class of
	processors. \lwmpi is designed from scratch and follow a two-tier
	approach to separate and self-contain the \mpi interface from the
	\os-dependent layer.
	%
	% Results
	Our experiments on the \mppa processor
	unveil that \lwmpi exposes a richer programming interface and
	achieves similar scalability for parallel and distributed problems, in
	comparison with the vendor-specific \api narrowed for the \mppa.
	%
	% Real contribution
	Thus, the improve in programmability for \lws and the implicit portability
	are our main contributions with \lwmpi.
