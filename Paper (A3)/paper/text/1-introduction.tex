\section{Introduction}
\label{sec:introduction}

	% Context: Lightweight Manycore Processors
	Lightweight manycore processors emerged to address demands on
	high-performance and energy efficiency~\cite{Francesquini2015}.
	On the one hand, to deliver high-performance and scalability, these
	processors rely on a distributed memory architecture and a rich
	\noc.  On the other hand, to achieve energy
	efficiency, they are built with simple low-power \mimd
	cores and \spms  with no hardware coherency
	support.  Moreover, they exploit heterogeneity
	by combining cores with different capabilities.
	Some industry-successful examples of \lws are the
	\mppa~\cite{DeDinechin2013-2} and the \epiphany~\cite{Olofsson2016}.

	% Motivation #1: Architectural Challenges
	However, while the aforementioned architectural features make lightweight
	manycores so interesting, they also introduce several challenges
	in software programmability.  For instance, the \textit{distributed
	memory architecture} requires a non-trivial software design where the
	data should be explicitly fetched from remote memories to local
	ones to be manipulated~\cite{Francesquini2015}.  Furthermore, the
	\textit{small amount of on-chip memory} demands software to
	explicitly tile the working data set into chunks and locally
	manipulate them one at a time~\cite{Souza2017}.
	Finally, the rich \noc exposes mechanisms for asynchronous programming
	to overlap communication with
	computation~\cite{Hascoet2017}; and hand-operated routing to
	guarantee uniform communication latencies.

	% Motivation #2: Software Development Support
	Currently, two approaches are employed to address programmability
	challenges in \lws: \oses~\cite{Kluge2014,
	Asmussen2016, Penna2019-3} and baremetal
	\runtimesys~\cite{Dinechin2013-1, Varghese2014, Richie2017}. The
	former is meant to bridge critical programmability gaps imposed by hardware
	intricacies.  The latter aims to expose a rich, performance-oriented
	programming environment, narrowed to the underlying architecture.
	While these two approaches are effective for some use cases, they
	have a significant duality drawback. Application development directly
	on top of \os interfaces yields to software portability, but the actual
	programming interface provided is
	complex and delay the software development process.  In contrast,
	baremetal and vendor-specific \runtimesys expose richer interfaces
	that accelerate the development process, but they exclusively concern to
	the software stack ecosystem of a specific lightweight manycore, resulting
	in non-portable applications.

	% The Paragraph: Problem, Hypothesis, Goals and Contributions
	In this work, we present a third proposition, trying to address the programmability and
	portability challenges of \lws, by combining both approaches: a lightweight
	implementation of the \mpi standard (named \lwmpi) on top of \nanvix,
	a \posix-compliant distributed \os that targets \lws~\cite{Penna2019-3}.
	%
	% Evaluation Methodology
	To assess \lwmpi with representative
	computing workloads, we carried out experiments with three applications extracted
	from the CAP Bench suite~\cite{Souza2017}. All experiments were executed on the
	\mppa, a baremetal lightweight manycore. Our results unveiled that \lwmpi delivers
	similar performance scalability when compared
	with a vendor-specific low-level \api for the \mppa, while exposing a richer
	programming interface.

	% Work Structure
	The remainder of this work is organized as follows.
	In Section~\ref{sec:related-work} we discuss related works. In
	Section~\ref{sec:proposal}, we present an architectural overview of \lwmpi.
	In Section~\ref{sec:evaluation-methodology}, we
	detail our evaluation methodology and present the \mppa. In
	Section~\ref{sec:experimental-results},
	we discuss the experimental results. Finally, in
	Section~\ref{sec:conclusions}, we draw our conclusions.
